\chapter{Related Work\label{cha:chapter2}}
Learned hyperspectral image compression, as explained before, is a developing field of study. While there are some papers published on this topic, there are some problems making it difficult to assess and compare the results of these studies, as will be illuminated further in this chapter. Since learned RGB image compression is a closely related field and much more widely researched, the studies done regarding this topic are also analysed.
\section{Hyperspectral image compression \label{sec:ch2hyperspectral}}
Most learned hyperspectral image compression papers use a CNN-based model architecture to reduce the dimensions of the input image \citep{kuester_1d-convolutional_2021}\citep{kuester_transferability_2022}\citep{la_grassa_hyperspectral_2022}. Another model proposed by Guo et.al. \citep{guo_learned_2021} uses the hyperprior architecture, originally developed by Ballé et.al. \citep{balle_end--end_2017}, which also uses convolutional layers in an ANN but combines them with an arithmetic coder to improve compression rate.\\
Some other models that will be explored later are an SVM-based model by Aidini et.al. \citep{aidini_hyperspectral_2019}, a GAN model by Deng et.al. \citep{deng_learning-based_2020} and a model using a simple multi-layer perceptron by Kumar et.al. \citep{leal-taixe_onboard_2019}.
\subsection{CNN-based architectures}
The CNN-based architectures can be split into two categories, the first being the models using two-dimensional convolutional layers to learn the spatial dependencies of the hyperspectral images \citep{la_grassa_hyperspectral_2022}. The second category are models using one-dimensional convolutional layers to learn the spectral dependencies of the input data \citep{kuester_1d-convolutional_2021}\citep{kuester_transferability_2022}. Prior to the release of this thesis there are no purely CNN-based papers using both the spatial and the spectral dependencies of hyperspectral images for compression. The model proposed by Guo et.al. \citep{guo_learned_2021} does use both spatial and spectral dependencies, it does however use a hyperprior architecture and not a purely CNN-based model.\\
As said before, comparing the results from these papers directly is difficult. The reason for this is that the models use different data sets since there is currently no accepted standard data set for hyperspectral imaging. Furthermore, the models use different compression rates. Since a higher compression rate also leads to a higher compression error for the same model as described by rate-distortion theory \citep{berger_rate-distortion_2003}, this makes it impossible to directly compare the results from papers that use both a different data set and different compression rates. 
\subsection{Other architectures}
\section{RGB image compression \label{sec:ch2rgb}}
\subsection{CNN-based architectures}
\subsection{Transformer-based architectures}